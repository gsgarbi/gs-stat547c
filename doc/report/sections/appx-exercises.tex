% !TEX root = ../main.tex

% Exercises section

\section{Exercises}

\textbf{Instructions:} Some exercises allow more than one possible answer - in those cases, an example of such an answer is provided as solution. We use the standard state notation described in section 2 unless stated otherwise. We use the measurable spaces E and F (or just E if E=F) and their sigma-algebras as we defined in part 1.

\begin{Exercise}
\begin{enumerate}[label=(\alph*)]
\item How would you define the identity kernel so that it behaves as expected with respect to the kernel operations we defined? 
\item Go back to section 1.2. Show that your definition is consistent with the properties of kernels given in 1.2.1.
\end{enumerate}
\end{Exercise}
\begin{Answer}
\begin{enumerate}[label=(\alph*)]
\item We need I to behave as following: $ KI = IK = K, If = f, \lambda I = \lambda $, for all kernel K, measurable function f, and measure $\lambda$. Therefore, according to the operation definitions we discussed, we define I as:
\begin{equation*}
I(x, B) = \begin{cases}
             0  & \text{if } x \not \in B \\
             1  & \text{if } x \in B
       \end{cases} \quad
\end{equation*} for all x in E and $B \in \calF$.


\item Our definition is consistent with the properties of kernels given in part 1 since:
\begin{enumerate}[label=(\roman*)]
\item for a fixed x, we have that $I = \delta_x(B)$, which is a measure.
\item for a fixed $B \in \calF$, $I = \mathbbm{1}_B(x)$, which is a measurable function.
\end{enumerate}
\end{enumerate}
\end{Answer}


\begin{Exercise}
\begin{enumerate}[label=(\alph*)]
\item Come up with a way to describe our state space that is different from our standard way (see part 2.2).
\item How would you represent the elements 123 and 221?
\item Using your notation, how many elements would your state space have? 
\end{enumerate}
\end{Exercise}
\begin{Answer}
\begin{enumerate}[label=(\alph*)]
\item One possibility is $i_1 ... i_n$, where $i_k$ counts the number of times element k is picked. Notice that, in this case, $\sum i_k = n$. 
\item 123 means we have a sample from each, so the representation would be 111. The representation of 221 would be 021.
\item We will not have as many as $3^3$ possibilities as with our standard notation because of our constraint that $\sum i_k = n$. Rather, the possible states are in a bijection with the number of permutations of ``2 bars and 3 balls": $O|O|O$. 
For instance, 111 and 021 are represented by $O|O|O$ and $|OO|O$, respectively. Think of this as 2 bars separating a total of 3 balls into 3 groups. 
Therefore, we have a total of $\frac{5!}{3!2!} = \binom{5}{3}$ elements.
For the general case, where an element is described as n entries adding up to n, we have $\binom{2n-1}{n}$ elements in our state space.
\end{enumerate}
\end{Answer}

\begin{Exercise}
Let $E = \{1, 2, ..., n\}$. Let a state $j = j_1...j_n$ be accessible from a state $i = i_1...i_n$ if $\{j_1, ..., j_n\} \subset E \setminus \{i_1, ..., i_n\} $. Define the kernel $K(i, \{j\})$ as: uniformly pick an accessible state j from state i. 
\begin{enumerate}[label=(\alph*)]
\item List (or draw) the states that communicate with 1441.
\item Argue that this chain is not irreducible.
\item If you were to simulate such a chain, how can you make it irreducible by making a small adaptation that fixes the problem you saw in part b?
\end{enumerate}
\end{Exercise}
\begin{Answer}
\begin{enumerate}[label=(\alph*)]
\item The values that do not appear in 1441 are 2 and 3, so the the states that communicate with 1441 are: 2222, 2223, 2232, 2233, 2322, 2323, 2333, 3333.
\item The chain is not irreducible because no state communicates with 1234.
\item The chain works fine, except for this problem with state 1234, so we need to find a way for other states to reach it and for them to be reached by it. One possibility is to always give a small possibility p for any state to access it and a uniform chance for it to access any of the other $4^4-1$ states. Any p is fine as long as the kernel stays a probability measure.
\end{enumerate}
\end{Answer}

\vspace{8cm}
\centerline{\textbf{\uppercase{PLEASE TURN PAGE FOR ANSWERS}}}

%2. Assume a sample i_1...i_n is greater than j_1...j_n if the number with algorithms composed by i_1, ..., i_n is greater than the one composed by j_1, ..., j_n. Example: 131 is greater than 122. Define your kernel with the following sampling step: "randomly pick a different, larger sample".


\clearpage
\shipoutAnswer
