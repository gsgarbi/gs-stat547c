% !TEX root = ../main.tex

% Exercises section

\section{Exercises}

We use the state notation described in part 2 when not stated otherwise.

\begin{enumerate}
\item Exercise:

\begin{enumerate}
\item How would you define the identity matrix in the context of kernels so that it behaves as the identity in multiplication? 
\item Show that your definition is consistent with the properties of kernels given in part 1.
\end{enumerate}




%2. Assume a sample i_1...i_n is greater than j_1...j_n if the number with algorithms composed by i_1, ..., i_n is greater than the one composed by j_1, ..., j_n. Example: 131 is greater than 122. Define your kernel with the following sampling step: "randomly pick a different, larger sample".

\item Exercise:

Let $E = \{1, 2, ..., n\}$. Let a state $j = j_1...j_n$ be accessible from a state $i = i_1...i_n$ if $\{j_1, ..., j_n\} \subset E \setminus \{i_1, ..., i_n\} $. Define the kernel $K(x, \{y\})$ as: uniformly pick an accessible state y from state x. 

a) List (or draw) the states that are connected to 112.

b) Argue that this chain is not irreducible.

c) If you were to simulate such a chain, how can you make irreducible by not changing the kernel "too much" (I mean, try not to create a whole new kernel)?

\textbf{Exercise:} 

a) Come up with a way to describe our state space different from the way described in part 2.

b) How would you represent the elements 123 and 221 from the example in section 2.1?

c) Using your notation, how many elements would your state space have? 

\textbf{Answer:} One possibility is $i_1 ... i_n$, where$ i_k$ counts the number of times element k is picked. Notice that, in this case, $\sum i_k = n$. 

\item Completely describe in Kernel notation for n = 3.

Prove theorem in the case of finite state spaces

\end{enumerate}


