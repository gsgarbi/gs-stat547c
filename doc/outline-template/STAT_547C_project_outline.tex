\documentclass[]{STAT_547C}
\usepackage{STAT_547C}

% NOTE: change the name and email address to your name in STAT_547C.sty

\usepackage{booktabs}
\usepackage{amsmath,amsthm,amssymb,amsfonts}

\usepackage[sorting=none,backend=biber,bibstyle=alphabetic,citestyle=alphabetic,giveninits=true,natbib=true]{biblatex}
\bibliography{../../ref/STAT_547C.bib} % add the title and location of your bibliography file

\usepackage{hyperref}

\begin{document}

\makeGenericHeader{Optimum Monte-Carlo Sampling Using Markov Chains \\ With Applications to Bootstrapping}{Project Outline}
\vspace{-2cm}


%%%%%%%%%%%%%%%%%%%
\section{Title}

The working title of my project is \emph{
Optimum Monte-Carlo Sampling Using Markov Chains with Applications to Bootstrapping
}.  

%%%%%%%%%%%%%%%%%%%
\section{Background}

Bootstrapping is a relatively well-understood and widely used re-sampling method basically consisting of independent re sampling steps with replacement. The paper by Peskun, P. H., however, makes us wonder if we can somehow improve (in the sense used by Peskun) the bootstrapping method by using a \textbf{dependent} re-sample step.

%%%%%%%%%%%%%%%%%%%
\section{Technical aspects}

The project will draw on technical aspects of the following areas: transition kernels, conditional expectations, sampling methods, statistical computing.


%%%%%%%%%%%%%%%%%%%
\section{Literature}

The key reference for this project is:

\begin{itemize}
\item \citet{Peskun, P. H. ``Optimum Monte-Carlo Sampling Using Markov Chains.'' Biometrika, vol. 60, no. 3, 1973, pp. 607-612. JSTOR}, \href{www.jstor.org/stable/2335011}{www.jstor.org/stable/2335011}
\end{itemize}


%%%%%%%%%%%%%%%%%%%
\section{Plan}

I will carry out this project with the following sequence of steps: 
\begin{enumerate}
	\item Formulate a mathematically precise interpretation of my research question
	\item Give a practical introduction to Markov Chains and bootstrapping
	\item Give special attention to formalization and notation for Markov Chains and bootstrapping within the scope of the project
  \item Map the main ideas presented by Peskun, P.H.
  \item Connect the paper by Peskun, P.H. to bootstrapping
  \item Provide exercises
  \item Provide an application on Python
\end{enumerate}


%%%%%%%%%%%%%%%%%%%
\section{Why I'm interested in this topic}

I like that the research question is not too hard to understand, but it seems to be an unanswered problem. It also covers some of the theory we are seeing in class and can help me practice working with a higher level of mathematical formality. I don't know if it counts, but I am curious to see what I can learn from this project.


%%%%%%%%%%%%%%%%%%%
\printbibliography
\section{References}
\begin{itemize}
\item \citet{Peskun, P. H. ``Optimum Monte-Carlo Sampling Using Markov Chains.'' Biometrika, vol. 60, no. 3, 1973, pp. 607-612. JSTOR}, \href{www.jstor.org/stable/2335011}{www.jstor.org/stable/2335011}
\end{itemize}


\end{document}

